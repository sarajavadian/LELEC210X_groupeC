\section{Practical part}
%
You are now ready to dive into a new embedded programming project. As you did for H2, git pull the following repository: \\
\url{https://forge.uclouvain.be/lelec2102-2103/h3b_code_package/}. \\
\\
We restart from the end of H2b, i.e. where you implemented the double buffering for sampling your signal with the ADC. What is new in this project is the content of the function called when you press the \textcolor{blue}{User blue button}: \\
%
\begin{lstlisting}
void HAL_GPIO_EXTI_Callback(uint16_t GPIO_Pin) {
	if ((GPIO_Pin == B1_Pin) & !bounce) {
		HAL_ADC_Start_DMA(&hadc1, (uint32_t *) ADCBuffer, 2 * SAMPLES_PER_MELVEC);
		HAL_TIM_Base_Start(&htim3);
		bounce = 1;
	}
}
\end{lstlisting}
%
Where you start the DMA each time the button is pressed. We also changed the content of \newline HAL$\_$ADC$\_$ConvCpltCallback for these two functions:
%
\begin{lstlisting}
void HAL_ADC_ConvHalfCpltCallback(ADC_HandleTypeDef *hadc) {
	Spectrogram_Format((q15_t *)ADCDblBuffer[0]);
	Spectrogram_Compute((q15_t *)ADCDblBuffer[0], mel_vectors[cur_melvec]);
	cur_melvec++;
	DEBUG_PRINT("Half DMA.\r\n");
}
\end{lstlisting}
\begin{lstlisting}
void HAL_ADC_ConvCpltCallback(ADC_HandleTypeDef *hadc) {
	Spectrogram_Format((q15_t *)ADCDblBuffer[1]);
	Spectrogram_Compute((q15_t *)ADCDblBuffer[1], mel_vectors[cur_melvec]);
	cur_melvec++;
	if (cur_melvec == N_MELVECS)
	{
		HAL_TIM_Base_Stop(&htim3);
		HAL_ADC_Stop_DMA(&hadc1);
		print_buffer(mel_vectors_flat, N_MELVECS * MELVEC_LENGTH);
		cur_melvec = 0;
	}
	bounce = 0;
	DEBUG_PRINT("All DMA.\r\n");
}
\end{lstlisting}
%
Which compute one feature vector each time the button is pressed. \\
You can find a new file \emph{spectrogram.c} which details exactly how we do the computations, which functions are used and why. We left lots of comments with some questions to ensure you follow it. Note you can vary the parameters in \emph{config.h}.  \\
\\

\begin{bclogo}[couleur = gray!20, arrondi = 0.2, logo=\bcinfo]{Casting data}
%
In the step $3.2$ of \emph{spectrogram.c}, you will observe this line of code:
%
\begin{lstlisting}
buf[i] = (q15_t) (((q31_t) buf_fft[i] << 15) / ((q31_t) vmax));
\end{lstlisting}
%
In this line, the (q15$\_$t) and two (q31$\_$t) are called \textbf{casting}. This consists in telling your compiler the format you wish the following variable to be stored in. Here, buf$\_$fft[i] is in q15$\_$t format, and the $\ll 15$ shift all its bits to the left $15$ times. The casting is necessary to get a temporary q31$\_$t variable and avoid overflowing your q15$\_t$ one.
%
\end{bclogo}
%
\begin{bclogo}[couleur = gray!20, arrondi = 0.2, logo=\bcinfo]{Hardcoding data}
The content of \emph{spectrogram$\_$tables.h} and \emph{spectrogram.h} has been hardcoded using the files in \textbf{Python2C$\_$conversion}. These notebooks could be useful for you during the second semester if you want to apply your feature vector computations on a perfect signal, i.e. an audio signal from the Dataset which has directly been hardcoded in C. This is optional, but you can already take a look at how this is done.
\end{bclogo}
%
When you read and try to understand the code (remember this is equivalent to what you did during H1 in Python), \textbf{evaluate the complexity of the different functions involved}, this will help you to identify the \emph{bottlenecks} (worst parts in a computational point of view). Each time you have a pipeline with different computations where the efficiency is critical, you should work on improving your bottlenecks. Think about it and \textbf{discuss with the TA's your ideas to improve} what we propose. \\
\\
We are not far from the true chain where we will still need to authenticate it then build a packet to be transmitted wirelessly. Removing the need to press the button is piece of cake. \\
\\
For now, the feature vector content is printed on the UART and displayed on the console (e.g. Putty). The function which reads the printed content (in \emph{hex}) is \textbf{uart$\_$reader.py}. It is fairly short. Understand what it does and add your functionalities from H2a to \textbf{apply a classification on the feature vector}. \\
\\
%
\noindent To summarize, all you have to do is:
\begin{enumerate}
    \item Understand the provided code.
    \item Check the consistency between a melspectrogram obtained only in simulation (as in H1) and obtained through the jack cable. Except for a random scaling, they should be almost identical.
    \item Modify \emph{uart$\_$reader.py} to test your classifier from H3a on the computed feature vector.
\end{enumerate}
